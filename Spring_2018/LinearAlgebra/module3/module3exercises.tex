\documentclass{article}
\usepackage{amsmath}
\usepackage{amssymb}

\title{Computational Linear Algebra, Module 3}
\author{Maya Shende}
\date{Due: February 21st, 2018}

\begin{document}
\maketitle

\begin{enumerate}
	
	\item 3 examples:
	\begin{enumerate}
		\item Netflix problem: grouping of users and their ratings is a natural grouping
		\item Text relevance: groupings can be word counts or probabilities of words 
		\item 3D to 2D coordinate system change: groupings are mapping of coordinates from 3d to 2d. 
	\end{enumerate}
	
	\item To verify by hand that addition produces the same result, we see that $(4,3) + (5,7) = (4+5,3+7) = (9,10)$. To prove that addition produces the same result regardless of coordinate system, I would take the coordinate system we are working in and map it to a (0,0)-centered coordinate system ((0,0,0)-centered in 3d), then do the addition in this known coordinate system and then map it back to the original coordinate system. In this theory, we add in the direction of the arrow, so b flipping the arrow, that would equate to subtraction, not addition. 
	
	\item Scalar multiplication doesn't change the ration between the x and y coordinates of a vctor, which is why the orientation (angle) doesn't change.
	
	\item See code.
	
	\item See code.
	
	\item See code.
	
	\item We have $a + 3b = 7.5$, so $a = 7.5-3b$. Now, we can plug this into the second equation and we have:
	\begin{eqnarray*}
		4a+2b &=& 10\\
		4(7.5-3b)+2b &=& 10\\
		30-12b+2b &=& 10\\
		30-10b&=&10\\
		20&=&10b\\
		b&=&2
	\end{eqnarray*}
	Now, plugging $b$ back into the first equation, we see that $a=1.5$. 
	
	\item No linear combination of vectors \textbf{u} and \textbf{v} will work because the two vectors are not linearly independent. First we have $a +3b =7.5$, which gives us that $a = 7.5-3b$ as above. Now, if we plugthis value of $a$ into the second equation we see:
	\begin{eqnarray*}
		2a+6b&=&10\\
		2(7.5-3b)+6b &=&10\\
		15-6b+6b &=& 10\\
		15 &=& 10
	\end{eqnarray*}
	which is not true. Therefore, there is no linear combination of the vectors to create the output vector. 
	
	\item First, we have $a +3b =4$, which gives us that $a = 4-3b$ as above. Now, if we plugthis value of $a$ into the second equation we see:
	\begin{eqnarray*}
		2a+6b&=&8\\
		2(4-3b) + 6b &=& 8\\
		8-6b+6b &=& 8\\
		8&=&8
	\end{eqnarray*}
	This means that there is not a unique solution; above we see that b can be anything, and thus a can be anything as long as it satisfies b. Geometrically, \textbf{u} and \textbf{v} are not linearly independent, so we can find lots of linear combinations of them to make z.
	
	\item The equations are:
	\begin{eqnarray*}
		a+3b-c &=& 7.5\\
		4a+2b+c &=&10
	\end{eqnarray*}
	Since there are more unknowns than equations, there are multiple possible solutions. A linear combination of \textbf{u} and \textbf{v} alone produces \textbf{z}, as seen in exercise 7.
	
	\item 
	\begin{eqnarray*}
		a + 4b + 3c &=& 1\\
		3a = b + 2c &=& 7\\
		a+6c &=& 8
	\end{eqnarray*}
	In these equations, I will first solve the third equation for $a$ and get $a = 8-6c$. Then, plugging this into the first and second equations I have the following two equations:
	\begin{eqnarray*}
		4b-3c &=& -7\\
		b-16c &=& -17
	\end{eqnarray*}
	So, if we solve the second equation for b and plug it into the first equation in this new set of two equations, we see that $c=1$ and $b=-1$. Then, plugging both of these into the original first third equation, we see that $a=2$. 
	
	\item See code. 
	
	\item If we try to solve these equations, we end up with an equation that $-80=-8$, which is not a true statement. All 3 of the original vectors lie in a single plane, and the result vector lies off of that plane, so no linear combination of the original three vectors will produce the result vector. We can see this graphically using the code.
	
	\item $A = 
	\begin{bmatrix}
		a_{11}	& a_{12} \\
		a_{21}	& a_{22} 
	\end{bmatrix}
	$
	\textbf{x} = 
	$
	\begin{bmatrix}
		x_1\\
		x_2
	\end{bmatrix}
	$
	So, $A$\textbf{x} = 
	$
	\begin{bmatrix}
		a_{11}x_1 + a_{12}x_2\\
		a_{21}x_1 + a_{22}x_2
	\end{bmatrix}
	$
	The dimensions of this are 2 x 1. 
	
	\item $y = 
	\begin{bmatrix}
		2	& -3 \\
		0	& 1 
	\end{bmatrix}
	\begin{bmatrix}
		2\\
		3
	\end{bmatrix}
	= 
	\begin{bmatrix}
		-5\\
		3
	\end{bmatrix}
	$. So, $z = 
	\begin{bmatrix}
		1	& 2\\
		0	&3
	\end{bmatrix}
	\begin{bmatrix}
		-5\\
		3
	\end{bmatrix}
	= 
	\begin{bmatrix}
		1\\
		-9
	\end{bmatrix}
	$

	\item See code. 
	
	\item $z = 
	\begin{bmatrix}
		2	& -1\\
		0	&-3
	\end{bmatrix}
	\begin{bmatrix}
		2\\
		3
	\end{bmatrix}
	= 
	\begin{bmatrix}
		1\\
		-9
	\end{bmatrix}
	$
	This is the same as \textbf{z} from exercise 15.
	
	\item $x = 
	\begin{bmatrix}
		\frac{1}{2}	& \frac{-1}{6}\\
		0	&\frac{-1}{3}
	\end{bmatrix}
	\begin{bmatrix}
		1\\
		-9
	\end{bmatrix}
	= 
	\begin{bmatrix}
		2\\
		3
	\end{bmatrix}
	$
	This is the original \textbf{x}
	
	\item $
	\begin{bmatrix}
		b_{11}	& b_{12}\\
		b_{21}	& b_{22}
	\end{bmatrix}
	\begin{bmatrix}
		a_{11}x_1 + a_{12}x_2\\
		a_{21}x_1 + a_{22}x_2
	\end{bmatrix}
	=
	\begin{bmatrix}
		b_{11}(a_{11}x_1 + a_{12}x_2) + b_{12}(a_{21}x_1 + a_{22}x_2)\\
		b_{21}(a_{11}x_1 + a_{12}x_2) + b_{22}(a_{21}x_1 + a_{22}x_2)	
	\end{bmatrix}
	$
	\item See code. 
	
	\item $y = 
	\begin{bmatrix}
		3	&2	&1\\
		-2	&3	&5\\
		0	&0	&3
	\end{bmatrix}
	\begin{bmatrix}
		1\\
		-1\\
		2
	\end{bmatrix}
	= 
	\begin{bmatrix}
		3\\	
		5\\
		6
	\end{bmatrix}
	$ . So, $z = 
	\begin{bmatrix}
		-4	&1	&0\\
		1	&0	&1\\
		3	&-2	&1
	\end{bmatrix}
	\begin{bmatrix}
		3\\
		5\\
		6
	\end{bmatrix}
	= 
	\begin{bmatrix}
		-7\\
		9\\
		5
	\end{bmatrix}
	$. $C = 
	\begin{bmatrix}
		-4	&1	&0\\
		1	&0	&1\\
		3	&-2	&1
	\end{bmatrix}
	\begin{bmatrix}
		3	&2	&1\\
		-2	&3	&5\\
		0	&0	&3
	\end{bmatrix}
	= 
	\begin{bmatrix}
		-14	&-5 	&1\\
		3	&2	&4\\
		13	&0	&-4
	\end{bmatrix}
	$. \\
	So, $Cx = 
	\begin{bmatrix}
		-14	&-5 	&1\\
		3	&2	&4\\
		13	&0	&-4
	\end{bmatrix}
	\begin{bmatrix}
		1\\
		-1\\
		2
	\end{bmatrix}
	= 
	\begin{bmatrix}
		-7\\
		9\\
		5
	\end{bmatrix}$
	
	\item See code. 
	
	\item $A = 
	\begin{bmatrix}
		a_{11}	&a_{12}  	&\dots	&a_{1n}\\
		a_{21} 	&a_{22}	&\dots	&a_{2n}\\
		\vdots	&\vdots	&\ddots	&\vdots\\
		a_{n1}	&a_{n2}	&\dots	&a_{nn}
	\end{bmatrix}
	$. And $B = 
	\begin{bmatrix}
		b_{11}	&b_{12}  	&\dots	&b_{1n}\\
		b_{21} 	&b_{22}	&\dots	&b_{2n}\\
		\vdots	&\vdots	&\ddots	&\vdots\\
		b_{n1}	&b_{n2}	&\dots	&b_{nn}
	\end{bmatrix}
	$. Now, if we multiply $AB$ and $BA$, we can see that corresponding terms in the two products are not the same. For example, $ab_{11} = a_{11}b_{11} + a_{12}b_{21} + \dots + a_{1n}b_{n1}$ and $ba_{11} = b_{11}a_{11} + b_{12}a_{21} + \dots + b_{1n}a_{n1}$, and these two sums are not equal. Since at least one term has been shown to be unequal between the two products, we can se ethat $AB \neq BA$. 
	
	\item $
	\begin{bmatrix}
		0	&1	&0\\
		-1	&2	&0
	\end{bmatrix}
	\begin{bmatrix}
		1	&1\\
		2	&2\\
		0	&-3
	\end{bmatrix}
	= 
	\begin{bmatrix}
		2	&2\\
		3	&3
	\end{bmatrix}
	$. $Ax = 
	\begin{bmatrix}
		0	&1	&0\\
		-1	&2	&0
	\end{bmatrix}
	\begin{bmatrix}
		2\\
		3\\
		4
	\end{bmatrix}
	= 
	\begin{bmatrix}
		3\\
		4
	\end{bmatrix}
	$
	
	\end{enumerate}

\end{document}